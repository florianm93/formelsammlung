\subsection{Integration und Differenzierung}

\subsection{Additionstheorme \textit{sin, cos, tan}}

\subsection{Rotationsk"orper}

\subsection{Logarithmus}

\subsection{Konvergenzkriterien f"ur Reihen}

\subsection{Binomischer Lehrsatz}

\subsection{Stetigkeit}

\subsection{Summen}

\subsection{Sonstige S"atze und Formeln}
	\subsubsection{Taylor- und MacLaurin-Reihe}
		\begin{align}
			&a_n \mbox{ sei } \frac{f^{(n)}(x-x_0)}{n!} \\
			&\Rightarrow p(x) = \sum_{n=0}^{\infty} a_n \cdot x^n \mbox{ (Taylor-Reihe, } x_0=0 \mbox{)} \\
			&\Rightarrow p(x) = \sum_{n=0}^{\infty} a_n \cdot (x-x_0)^n \mbox{ (MacLaurin-Reihe)}
		\end{align}

	\subsubsection{Nullstellensatz}
		\begin{align}
			f(a) \cdot f(b) < 0 \exists x^* \in (a, b]: f(x^*) = 0 
		\end{align}

\subsection{Komplexe Zahlen}
